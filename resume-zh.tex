% !TEX TS-program = xelatex
% !TEX encoding = UTF-8 Unicode
% !Mode:: "TeX:UTF-8"

\documentclass{resume}
\usepackage{zh_CN-Adobefonts_external} % Simplified Chinese Support using external fonts (./fonts/zh_CN-Adobe/)
% \usepackage{NotoSansSC_external}
% \usepackage{NotoSerifCJKsc_external}
% \usepackage{zh_CN-Adobefonts_internal} % Simplified Chinese Support using system fonts
\usepackage{linespacing_fix} % disable extra space before next section
\usepackage{cite}

\begin{document}
\pagenumbering{gobble} % suppress displaying page number

\name{帅天强}

\basicInfo{
  \email{i@shuaitq.com} \textperiodcentered\ 
  \phone{(+86) 151-1537-2806} \textperiodcentered\ 
  \github[shuaitq]{https://github.com/shuaitq} \textperiodcentered\ 
  \homepage{http://shuaitq.com}}

\section{\texorpdfstring{\faGraduationCap\ 教育背景}{教育背景}}
\datedsubsection{\textbf{哈尔滨工业大学}}{2016 -- 2020}
\textit{学士}\ 计算机科学与技术

\section{\texorpdfstring{\faCogs\ 专业技能}{专业技能}}
% increase linespacing [parsep=0.5ex]
\begin{itemize}[parsep=0.5ex]
  \item 能熟练使用C++, Java进行开发,对其他语言有所了解,能短时间学习使用新语言
  \item 熟悉计算机体系结构,编写过RISC-V, MIPS指令集CPU,了解CPU缓存、原子操作、内存屏障原理
  \item 熟练使用Linux系统,能在Linux环境进行日常开发
%  \item 对计算机底层技术和Web技术非常感兴趣
%  \item 参加过ACM竞赛,掌握常用算法和数据结构
  \item 能够使用Spring框架进行Web开发,使用过Hive, Flink, Kafka, Redis等开源组件
%  \item 熟练掌握算法、数据结构、操作系统、硬件结构等知识
  \item 熟练使用Git进行版本管理,使用GitLab进行团队协作,了解现代软件开发流程
\end{itemize}

\section{\texorpdfstring{\faUsers\ 工作经历}{工作经历}}
% \datedsubsection{\textbf{虚拟网络安全互联}, 学校实验室}{2017年9月 -- 2018年1月}
% \role{C, Linux}{使用网络代理的形式,在上层应用不需要修改的情况下保护应用流量}
% \begin{itemiz%e}
%   \item 使用epoll技术实现服务端多路IO复用
%   \item 使用Google的开源加密库libsodium实现多种加密方式支持
% \end{itemize}

% \datedsubsection{\textbf{安全WiFi App}}{2018年1月 -- 2018年2月}
% \role{Java, Android}{实验室项目,根据当前网络和授信列表,在不安全网络自动开启加密连接,保护上网安全}
% \begin{itemize}
%   \item 按照美工设计,实现了界面效果和界面逻辑
%   \item 对接学长加密连接服务,能够正确处理网络切换事件
% \end{itemize}

\datedsubsection{\textbf{拼多多}, 上海}{2021年7月 -- 2022年3月}
\role{C++, Flink}{\textbf{DSP组}\quad 服务端研发工程师}
\begin{itemize}
  \item 负责组内K8s集群Spring管控服务维护开发,使用Flink处理数据供在线服务使用。
  \item 负责C++特征存储服务开发和运维,管理在线服务查询特征和特征部署问题。
  \item 使用Valgrind分析C++特征构建库内存泄漏问题并解决。
\end{itemize}

\datedsubsection{\textbf{拼多多}, 上海}{2020年7月 -- 2021年6月}
\role{Java, Spring, Hive}{\textbf{广告投放组}\quad 服务端研发工程师}
\begin{itemize}
  \item 负责站外落地页后端Spring服务维护,完成产品需求和对接前端开发。
  \item 使用Hive离线处理广告埋点数据,导入报表系统展示成报表。
  \item 服务接入AB测试系统,能自动创建关闭AB实验,用Hive处理埋点数据导回AB系统做展示分析。
  \item 开发智能样式系统,根据埋点数据采用不同策略自动决策落地页所展示样式。
\end{itemize}

\datedsubsection{\textbf{拼多多}, 上海}{2019年7月 -- 2019年8月}
\role{Java, Spring}{\textbf{广告投放组}\quad 服务端研发实习生}
\begin{itemize}
  \item 熟悉Java和Spring框架,完成新手项目编写。
  \item 根据阿里汇川广告API接入文档,完成阿里汇川渠道广告创建接入,能够批量创建图片视频广告。
  \item 分析Java中使用包装类型和使用基础类型性能和内存占用的差异。
\end{itemize}

% Reference Test
%\datedsubsection{\textbf{Paper Title\cite{zaharia2012resilient}}}{May. 2015}
%An xxx optimized for xxx\cite{verma2015large}
%\begin{itemize}
%  \item main contribution
%\end{itemize}

\section{\texorpdfstring{\faGithubAlt\ 个人项目}{个人项目}}
\datedsubsection{\textbf{MoonLight}}{\url{https://github.com/shuaitq/MoonLight}}
\role{C++}{软件光线追踪渲染器}
\begin{itemize}
  \item 支持三种相机模型:透视相机、鱼眼相机、正交相机。
  \item 支持三种材质:玻璃、镜面、磨砂材质,能够正确渲染材质的粗糙度。
\end{itemize}

\datedsubsection{\textbf{MIPS-CPU}}{\url{https://github.com/shuaitq/MIPS-CPU}}
\role{Verilog}{MIPS32 Release 1指令集五级流水线CPU}
\begin{itemize}
  \item 支持69条MIPS指令,大多数指令可以在一个时钟周期内完成。
  \item 五级整数流水线,分别是:取指、译码、执行、访存、回写。
\end{itemize}

% \datedsubsection{\textbf{EmbeddedCD}}{\url{https://github.com/shuaitq/EmbeddedCD}}
% \role{Python3, Raspberry Pi}{使用树莓派制作的温湿度监控}
% \begin{itemize}
%   \item 使用DHT22温湿度传感器读取环境温湿度,并写入sqlite3数据库存储
%   \item 使用Python3 Flask框架实现后端,前端使用ECharts绘制24小时温湿度曲线,JS定时更新页面
%   \item 使用PyQt实现桌面端,使用PyQtChart绘制最近5分钟温湿度曲线,实时更新
% \end{itemize}

\datedsubsection{\textbf{Aurora}}{\url{https://github.com/shuaitq/Aurora}}
\role{C++}{软件光栅化渲染器}
\begin{itemize}
%  \item 使用json格式定义场景、相机、灯光等参数
  \item 采用obj格式模型,ppm格式贴图,支持双线性过滤。
  \item 支持方向光和点光源两种光源。
  \item 采用Z-Buffer保证渲染正确的顺序,支持背面消隐和三角形剔除。
\end{itemize}

% \section{\texorpdfstring{\faHeartO\ 获奖情况}{获奖情况}}
% \datedline{\textit{铜奖}, 第八届山东省ACM大学生程序设计竞赛}{2017年5月}
% \datedline{\textit{提高组二等奖}, 湖南省Noip2015}{2015年11月}
% \datedline{\textit{提高组二等奖}, 湖南省Noip2014}{2014年11月}
% \datedline{\textit{提高组三等奖}, 湖南省Noip2013}{2013年11月}

% \section{\texorpdfstring{\faInfo\ 其他}{其他}}
% increase linespacing [parsep=0.5ex]
% \begin{itemize}[parsep=0.5ex]
%  \item 博客:\url{http://shuaitq.com}
%  \item 获取此简历的最新版本:\url{https://github.com/shuaitq/resume}
%  \item 热爱学习新的技术和知识,乐于交朋友
%  \item 语言: 英语 - 四级
% \end{itemize}

%% Reference
%\newpage
%\bibliographystyle{IEEETran}
%\bibliography{mycite}
\end{document}
