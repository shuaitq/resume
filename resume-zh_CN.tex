% !TEX TS-program = xelatex
% !TEX encoding = UTF-8 Unicode
% !Mode:: "TeX:UTF-8"

\documentclass{resume}
\usepackage{zh_CN-Adobefonts_external} % Simplified Chinese Support using external fonts (./fonts/zh_CN-Adobe/)
% \usepackage{NotoSansSC_external}
% \usepackage{NotoSerifCJKsc_external}
% \usepackage{zh_CN-Adobefonts_internal} % Simplified Chinese Support using system fonts
\usepackage{linespacing_fix} % disable extra space before next section
\usepackage{cite}

\begin{document}
\pagenumbering{gobble} % suppress displaying page number

\name{帅天强}

\basicInfo{
  \email{i@shuaitq.com} \textperiodcentered\ 
  \phone{(+86) 151-1537-2806} \textperiodcentered\ 
  \github[shuaitq]{https://github.com/shuaitq}}
 
\section{\faGraduationCap\  教育背景}
\datedsubsection{\textbf{哈尔滨工业大学(威海)}, 山东}{2016 -- 至今}
\textit{学士}\ 计算机科学与技术, 预计 2020 年 6 月毕业

\section{\faUsers\ 项目经历}
\datedsubsection{\textbf{分布式科学上网姿势}}{2017年9月 -- 2018年1月}
\role{C, Linux}{实验室项目}
分布式负载均衡科学上网姿势, https://github.com/cyfdecyf/cow
\begin{itemize}
  \item 修复了连接未正常关闭导致文件描述符耗尽的 bug
  \item 使用Chord 哈希 URL, 实现稳定可靠地分流
  \item xxx (尽量使用量化的客观结果)
\end{itemize}

\datedsubsection{\textbf{分布式科学上网姿势}}{2018年1月 -- 2018年2月}
\role{Java, Android}{实验室项目}
分布式负载均衡科学上网姿势, https://github.com/cyfdecyf/cow
\begin{itemize}
  \item 修复了连接未正常关闭导致文件描述符耗尽的 bug
  \item 使用Chord 哈希 URL, 实现稳定可靠地分流
  \item xxx (尽量使用量化的客观结果)
\end{itemize}

% Reference Test
%\datedsubsection{\textbf{Paper Title\cite{zaharia2012resilient}}}{May. 2015}
%An xxx optimized for xxx\cite{verma2015large}
%\begin{itemize}
%  \item main contribution
%\end{itemize}

\section{\faGithubAlt\ 个人项目}
\datedsubsection{\textbf{Aurora}}{\url{https://github.com/shuaitq/Aurora}}
\role{C++}{CPU计算光栅化渲染器}
\begin{itemize}
  \item 支持使用json格式定义场景、相机、灯光等参数。
  \item 支持obj格式模型,支持ppm格式图片贴图,采用二次线性差值进行采样,使用平铺纹理坐标寻址。
  \item 支持方向光和点光源两种光源。
  \item 采用Z-Buffer保证渲染正确的顺序,支持背面消影和三角形剔除
\end{itemize}

\datedsubsection{\textbf{MIPS-CPU}}{\url{https://github.com/shuaitq/MIPS-CPU}}
\role{Verilog}{基于MIPS32 Release 1指令集的五级流水线CPU。}
\begin{itemize}
  \item 五级整数流水线,分别是:取指、译码、执行、访存、回写。
  \item 哈佛结构,分开的指令、数据接口。
  \item 支持延迟转移。
  \item 兼容MIPS32指令集架构,支持MIPS32指令集中的大部分指令。
  \item 大多数指令可以在一个时钟周期内完成。
\end{itemize}

\datedsubsection{\textbf{Game-of-Life}}{\url{https://github.com/shuaitq/Game-of-Life}}
\role{C++}{康威生命游戏模拟器}
\begin{itemize}
  \item 支持通用RLE格式模式文件。
  \item 能够通过简单的命令行参数设置模拟用的模式文件和模拟次数。
\end{itemize}

\section{\faCogs\ IT 技能}
% increase linespacing [parsep=0.5ex]
\begin{itemize}[parsep=0.5ex]
  \item 编程语言: C++ = C > Java = Python = Go > Haskell > Rust
  \item 平台: Linux
  \item 开发: xxx
\end{itemize}

\section{\faHeartO\ 获奖情况}
\datedline{\textit{第一名}, xxx 比赛}{2013 年6 月}
\datedline{其他奖项}{2015}

\section{\faInfo\ 其他}
% increase linespacing [parsep=0.5ex]
\begin{itemize}[parsep=0.5ex]
  \item 技术博客: {http://shuaitq.com}
  \item GitHub: {https://github.com/shuaitq}
  \item 语言: 英语 - 熟练(TOEFL xxx)
\end{itemize}

%% Reference
%\newpage
%\bibliographystyle{IEEETran}
%\bibliography{mycite}
\end{document}
