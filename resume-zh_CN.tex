% !TEX TS-program = xelatex
% !TEX encoding = UTF-8 Unicode
% !Mode:: "TeX:UTF-8"

\documentclass{resume}
\usepackage{zh_CN-Adobefonts_external} % Simplified Chinese Support using external fonts (./fonts/zh_CN-Adobe/)
% \usepackage{NotoSansSC_external}
% \usepackage{NotoSerifCJKsc_external}
% \usepackage{zh_CN-Adobefonts_internal} % Simplified Chinese Support using system fonts
\usepackage{linespacing_fix} % disable extra space before next section
\usepackage{cite}

\begin{document}
\pagenumbering{gobble} % suppress displaying page number

\name{帅天强}

\basicInfo{
  \email{i@shuaitq.com} \textperiodcentered\ 
  \phone{(+86) 151-1537-2806} \textperiodcentered\ 
  \github[shuaitq]{https://github.com/shuaitq}}
 
\section{\faGraduationCap\  教育背景}
\datedsubsection{\textbf{哈尔滨工业大学(威海)}, 山东}{2016 -- 至今}
\textit{学士}\ 计算机科学与技术, 预计 2020 年 6 月毕业

\section{\faCogs\ IT 技能}
% increase linespacing [parsep=0.5ex]
\begin{itemize}[parsep=0.5ex]
  \item 编程语言: C++ with C++11 = C > Java = Python = Go > Haskell > Rust
  \item 日常使用Linux开发,能熟练使用Linux
  \item 熟练掌握算法、数据结构、操作系统、计算机网络、计算机组成原理
  \item 熟悉数据库、编译原理
  \item 热爱学习新的技术和知识
\end{itemize}

\section{\faUsers\ 项目经历}
\datedsubsection{\textbf{P2P虚拟网络安全互联}}{2017年9月 -- 2018年1月}
\role{C, Linux}{实验室项目,基于对等网络设计,能穿透部分种类NAT,支持多种加密算法,实现跨网络安全互联。}
\begin{itemize}
  \item 参与通信协议设计。
  \item 与学姐合作实现出第一版可用程序。
  \item 参与多平台复用库设计。
\end{itemize}

\datedsubsection{\textbf{安全WiFi App}}{2018年1月 -- 2018年2月}
\role{Java, Android}{实验室项目,根据当前网络和授信列表,在不安全网络自动开启加密连接,保护上网安全。}
\begin{itemize}
  \item 按照美工设计,实现了整个界面和界面逻辑。
  \item 对接学长加密连接服务。
\end{itemize}

% Reference Test
%\datedsubsection{\textbf{Paper Title\cite{zaharia2012resilient}}}{May. 2015}
%An xxx optimized for xxx\cite{verma2015large}
%\begin{itemize}
%  \item main contribution
%\end{itemize}

\section{\faGithubAlt\ 个人项目}
\datedsubsection{\textbf{MoonLight}}{\url{https://github.com/shuaitq/MoonLight}}
\role{C++}{CPU计算蒙特卡洛方法光线追踪渲染器}
\begin{itemize}
  \item 支持三种相机模型,透视相机、鱼眼相机、正交相机。
  \item 支持三种材质、玻璃、镜面、磨砂材质。
  \item 能够正确的渲染材质的粗糙度。
\end{itemize}

\datedsubsection{\textbf{MIPS-CPU}}{\url{https://github.com/shuaitq/MIPS-CPU}}
\role{Verilog}{MIPS32 Release 1指令集的五级流水线CPU}
\begin{itemize}
  \item 五级整数流水线,分别是:取指、译码、执行、访存、回写。
  \item 哈佛结构,分开的指令、数据接口。
  \item 支持延迟转移,大多数指令可以在一个时钟周期内完成。
\end{itemize}

\datedsubsection{\textbf{Aurora}}{\url{https://github.com/shuaitq/Aurora}}
\role{C++}{CPU计算光栅化渲染器}
\begin{itemize}
  \item 采用obj格式模型,ppm格式图片贴图,二次线性差值进行采样,平铺纹理坐标寻址。
  \item 支持方向光和点光源两种光源。
  \item 采用Z-Buffer保证渲染正确的顺序,支持背面消影和三角形剔除。
\end{itemize}

\section{\faHeartO\ 获奖情况}
\datedline{\textit{三等奖}, 第八届山东省ACM大学生程序设计竞赛}{2017年5月}
\datedline{\textit{提高组二等奖}, 湖南省Noip2015}{2015年11月}
\datedline{\textit{提高组二等奖}, 湖南省Noip2014}{2014年11月}
\datedline{\textit{提高组三等奖}, 湖南省Noip2013}{2013年11月}

\section{\faInfo\ 其他}
% increase linespacing [parsep=0.5ex]
\begin{itemize}[parsep=0.5ex]
  \item 博客: \url{http://shuaitq.com}
  \item 语言: 英语 - 四级
\end{itemize}

%% Reference
%\newpage
%\bibliographystyle{IEEETran}
%\bibliography{mycite}
\end{document}
